\chapter{Hoe is Appsemble’s UX ten opzichten van de concurrentie}
Om een analyse te doen van de concurrentie van Appsemble is het van belang om eerst vast te stellen wat de concurrentie is. Hier is binnen Appsemble al een aantal keer naar gekeken.\\
\newline
%
De voornaamste concurrentie is de website Appmachine, deze bieden ongeveer dezelfde dienst met een aantal verschillen waaronder dat het maken van apps met Appmachine \$49 per maand kost. \\
\newline
%
Bij een zoektocht op het internet komen ook een hoop websites tevoorschijn waar je al dan niet gratis je app kan maken, deze apps zijn echter allemaal van uiterst twijfelachtige kwaliteit. Deze websites zullen dus niet worden meegenomen in de analyse.\\
\newline
%
Er zijn nog een aantal andere websites die een dienst aanbieden die verschilt van Appsemble en 
Appmachine ( waar bovengenoemde werken met voorgemaakte componenten, bieden deze websites kleine onderdelen die bij elkaar gevoegd worden om pagina’s en apps te maken) we kunnen er echter nog steeds naar kijken hoe zij hun UX aanpakken, deze websites zullen dus worden meegenomen in de analyse.\\
\newline
De websites die worden meegenomen in de analyse zijn de volgende:
\begin{itemize}
	\item \href{http://www.appmachine.com/}{Appmachine} \cite{appMachine}
	
	\item \href{https://appery.io/}{Appery.io} \cite{appery}
	
	\item \href{http://www.goodbarber.com/}{Good Barber} \cite{goodBarber}
	
	\item \href{https://www.biznessapps.com/}{BiznessAPPS} \cite{biznessApps}
\end{itemize}
\emph{Alle websites zijn bekeken op een Ubuntu computer in Chromium, 52 \& Firefox 49.}\\
\newline
\section{Wat doet de concurrentie goed?} 
\begin{itemize}
	\item Concurrentie bied de mogelijkheid om een template te gebruiken en zo een app te maken waar al een aantal onderdelen in staan, zodat je sneller aan de slag kan als je zo’n soort app wil maken. Hierdoor wordt het als gebruiker een stuk makkelijker om een app in elkaar te zetten.
	
	\item Good Barber \& BiznessApps bieden een visuele indicatie van wat er voor een app nodig is om gepubliceerd te worden in de stroes, dit bied de gebruiker meer context in waar de informatie voor nodig is en maakt het makkelijker om deze in te vullen.
	
	\item Een aantal van de concurrenten heeft bij het uitkiezen van een component een korte uitleg van het component en een voorbeeld van hoe het eruit gaat zien, dit helpt de gebruiker om het juiste onderdeel te kiezen.
	
	\item Sommige concurrenten hebben een mogelijkheid om je app een aantal styles te geven, zodat je dus snel een app in elkaar kan zetten die eruit ziet hoe je wil.
	
	\item Good Barber maakt gebruik van een sidebar waar de stappen op staan die je moet uitvoeren om je eigen app te maken, de stappen hoeven echter niet perce gevolgd te worden. Dit is een vorm van onboarding die niet storend is voor ervaren gebruikers van de app omdat ze het ook kunnen gebruiken als een manier om te navigeren
	
	\item BiznessApps bied de mogelijkheid om componenten toe te voegen, maar heb je de kennis en middelen dan kan je ook je eigen design maken.
\end{itemize}
\section{Wat doet de concurrentie niet zo goed?} 
\begin{itemize}
	\item Appery.io werkt niet goed op Chromium / Chrome, op Firefox ziet alles er prima uit maar in Chromium / Chrome werkt alles niet goed. Ondanks dat Chrome wel tot de aanbevolen browsers behoord.
	
	\item Voor een gewone gebruiker zou Appery.io niet te gebruiken zijn, er zitten teveel technische termen in die voor een gewone gebruiker niet te gebruiken zijn. Tijdens het maken van een app en een component kwamen de volgende technische termen voorbij: Ionic, Bootstrap, JQuery Mobile, CSS, Modal, API Express Service, Scope, Binding ng-init \& Class . Dit is prima voor Developers die ervaring hebben met Javascript en het maken van websites maar voor een gewone gebruiker zal dit extreem afschrikwekkend werken.
	
	\item Veel van de concurrentie bied niet de mogelijkheid om kleuren binnen een pallete te kiezen, het is of het thema volgen of je eigen kleuren uitkiezen.
	
	\item Good Barber app is gecrasht tijdens het aanmaken van een app en werkt sindsdien niet meer zonder de user hier een melding van te geven. Bij het maken van een nieuwe app verdween het probleem
	
	\item Good Barber bied de mogelijkheid om extra fonts toe te voegen vanuit Google Fonts, dit is een goede feature maar het is niet mogelijk om te zoeken naar een font en er is geen manier om bijvoorbeeld naar de fonts te gaan die beginnen met een K.
	
	\item Appmachine geeft geen duidelijke uitleg wat er mis is met een block als daar een fout in zit, hierdoor moet de gebruiker gaancontent... zoeken naar wat de fout is.
\end{itemize}
\section{Swot Appsemble}
\subsection{Strenght}
\begin{itemize}
	\item Gratis
	
	\item Het platform is te proberen zonder te registreren.
\end{itemize}
\subsection{Weakness}
\begin{itemize}
	\item Niet zo visueel aantrekkelijk als de directe concurrentie.
	
	\item De website is heeft een aantal lastige technische termen
	
	\item De website werkt niet intuitief
	
\end{itemize}
\subsection{Opportunity}
\begin{itemize}
	\item De technology achter Appsemble lijkt een van de betere te zijn/
	
	\item Slechts een klein deel van de concurrentie bied de mogelijkheid om een website en een app te maken.
	
	\item Concurrentie die niet erg goed werkt of er goed uit ziet, krijgt erg veel downloads. Dit betekent dat er een markt is voor dit type dienst.
	
	\item Deel van de concurrentie komt complex over voor een leek.
\end{itemize}
\subsection{Threat}
\begin{itemize}
	\item Veel concurrentie
	
	\item Geen unique selling points dat voor de gemiddelede consument interresant is.
\end{itemize}





